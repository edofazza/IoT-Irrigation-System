\section{CoAP Network}
AAa

\subsection{Temperature Sensor}

\subsection{Soil Moisture Sensor}

\subsection{Rain Sensor}
Rain sensor or rain switch is a switching device activated by rainfall. It is used for water conservation since it is connected to the automatic irrigation system, which will cause the system to shut down in the event of rainfall in order to do not waste water and to reduce energy consumption.

\subsubsection{Resource}
The only resource provided by the rain sensor is a value indicating if it is raining or not, named \textbf{isRaining} and stored as a boolean. Since we are only interested when the status of the variable change, we opt to use the observable pattern provided by CoAP in order to minimize the number of interactions with the sensor.

The only possible action is the \textbf{GET} method, which will respond with a text saying \textit{"raining"} or \textit{"not raining"} based on the status of \textit{isRaining}.

\subsubsection{Data Generation}
Data is generated every \textit{CLOCK\_SECOND} in order to have a rapid simulation. The value of \textbf{isRaining} flips (i.e., if it was indicating raining it turns to not raining, and vice versa) with a probability of 10\%. This is done in the textiti{rain\_event\_handler} function in the following way:

\begin{lstlisting}
static void rain_event_handler(void)
{
    srand(time(NULL));
    int random = rand() % 10; // generate 0, 1, ..., 9
    
    if (random == 0) // 10% probability of changing the value
        new_isRaining = !isRaining;

    // if not equal, notify
    if (new_isRaining != isRaining)
        coap_notify_observers(&rain_sensor);
}
\end{lstlisting}

In case the value changes, this is notified to all the subscribers.

\subsection{Tap Actuator}
