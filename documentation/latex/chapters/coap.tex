\section{CoAP Network}
AAa




\subsection{Temperature Sensor}
The temperature sensor measures the local temperature in Celsius (at the Collector level will be given the possibility to display the results in Fahrenheit, see the Collector chapter for further details). The goal of this sensors is to quantify and schedule the water provisioning.

\subsubsection{Resources}
The temperature sensor exposes two resources: the \textit{temperature\_sensor} and the \textit{temperature\_switch} resources.

The \textbf{temperature\_sensor} resource is an observable resource that provides to the clients the temperature acquired by the sensor. The resource not only provides the mere temperature to the clients, but it informs if the temperature is lower or greater than a certain threshold. Hence, the sensor exposes a  \textit{PUT} method, in order to set up the lower or the upper bound for the temperature.

The change of the bounds is done at step, the user will specify the threshold that he/she wants to change through the CLI: upper or lower. At the server side the request will be processed checking if the value arrived is consistent (e.g., the new value for the lower bound is not greater than the upper bound actual value), after those controls the parameter is updated.

The \textbf{temperature\_switch} resource is connected to the \textit{isActive} boolean variable, which indicates if the sensor is operating or not. This is done for turning off the temperature sensor when it is raining in order to save energy, since we do not perform any analysis for irrigating when the weather does that for us. For the reason that we want to change the status of the resources based on the rain sensor, it is implemented a \textit{PUT} method for changing the value of the \textit{isActive} variable.

\subsubsection{Data Generation}
Data is generated every \textit{CLOCK\_SECOND} in order to have a rapid simulation. The value for the temperature is updated using the following algorithm:

\begin{lstlisting}
static void temperature_event_handler(void)
{
    if (!isActive) {
        return; // DOES NOTHING SINCE IT IS TURNED OFF
    }
    
    // extimate new temperature
    srand(time(NULL));
    int new_temp;
    int random = rand() % 4; // generate 0, 1, 2, 3
    
    if (random == 0) // 25% of changing the value
        if (random < 2) // decrease
            temperature -= VARIATION;
        else // increase
            temperature += VARIATION;

    // if not equal
    if (new_temp != temperature)
    {
        temperature = new_temp;
        coap_notify_observers(&temperature_sensor);
    }
}
\end{lstlisting}



\subsection{Soil Moisture Sensor}










\subsection{Rain Sensor}
Rain sensor or rain switch is a switching device activated by rainfall. It is used for water conservation since it is connected to the automatic irrigation system, which will cause the system to shut down in the event of rainfall in order to do not waste water and to reduce energy consumption.

\subsubsection{Resource}
The only resource provided by the rain sensor is a value indicating if it is raining or not, named \textbf{isRaining} and stored as a boolean. Since we are only interested when the status of the variable change, we opt to use the observable pattern provided by CoAP in order to minimize the number of interactions with the sensor.

The only possible action is the \textbf{GET} method, which will respond with a text saying \textit{"raining"} or \textit{"not raining"} based on the status of \textit{isRaining}.

\subsubsection{Data Generation}
Data is generated every \textit{CLOCK\_SECOND} in order to have a rapid simulation. The value of \textbf{isRaining} flips (i.e., if it was indicating raining it turns to not raining, and vice versa) with a probability of 10\%. This is done in the \textit{rain\_event\_handler} function in the following way:

\begin{lstlisting}
static void rain_event_handler(void)
{
    srand(time(NULL));
    int random = rand() % 10; // generate 0, 1, ..., 9
    
    if (random == 0) // 10% probability of changing the value
        new_isRaining = !isRaining;

    // if not equal, notify
    if (new_isRaining != isRaining)
        coap_notify_observers(&rain_sensor);
}
\end{lstlisting}

In case the value changes, this is notified to all the subscribers.








\subsection{Tap Actuator}
