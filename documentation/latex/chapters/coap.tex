\section{CoAP Network}
AAa




\subsection{Temperature Sensor}
The temperature sensor measures the local temperature in Celsius (at the Collector level will be given the possibility to display the results in Fahrenheit, see the Collector chapter for further details). The goal of this sensors is to quantify and schedule the water provisioning.

\subsubsection{Resources}
The temperature sensor exposes two resources: the \textit{temperature\_sensor} and the \textit{temperature\_switch} resources.

The \textbf{temperature\_sensor} resource is an observable resource that provides to the clients the temperature acquired by the sensor. The resource not only provides the mere temperature to the clients, but it informs if the temperature is lower or greater than a certain threshold. Hence, the sensor exposes a  \textit{PUT} method, in order to set up the lower or the upper bound for the temperature.

The change of the bounds is done at step, the user will specify the threshold that he/she wants to change through the CLI: upper or lower. At the server side the request will be processed checking if the value arrived is consistent (e.g., the new value for the lower bound is not greater than the upper bound actual value), after those controls the parameter is updated.

The \textbf{temperature\_switch} resource is connected to the \textit{isActive} boolean variable, which indicates if the sensor is operating or not. This is done for turning off the temperature sensor when it is raining in order to save energy, since we do not perform any analysis for irrigating when the weather does that for us. For the reason that we want to change the status of the resources based on the rain sensor, it is implemented a \textit{PUT} method for changing the value of the \textit{isActive} variable.

\subsubsection{Data Generation}
Data is generated every \textit{CLOCK\_SECOND} in order to have a rapid simulation. The value for the temperature is updated using the following algorithm:

\begin{lstlisting}
static void temperature_event_handler(void)
{
    if (!isActive) {
        return; // DOES NOTHING SINCE IT IS TURNED OFF
    }
    
    // extimate new temperature
    srand(time(NULL));
    int new_temp;
    int random = rand() % 4; // generate 0, 1, 2, 3
    
    if (random == 0) // 25% of changing the value
        if (random < 2) // decrease
            new_temp -= VARIATION;
        else // increase
            new_temp += VARIATION;

    // if not equal
    if (new_temp != temperature)
    {
        temperature = new_temp;
        coap_notify_observers(&temperature_sensor);
    }
}
\end{lstlisting}



\subsection{Soil Moisture Sensor}
Soil moisture sensors measure the water content in the soil and can be used to estimate the amount of stored water in the soil horizon. Soil moisture sensors do not measure water in the soil directly. Instead, they measure changes in some other soil property that is related to water content in a predictable way. Checking the different technologies used for measure soil moisture content, we decide to exploit the \textit{soil water potential}\footnote{\textit{Soil water potential} or \textit{soil moisture tension} is a measurement of how tightly water clings to the soil and is expressed in units of pressure called bars. Generally, the drier the soil, the greater the soil water potential and the harder a plant must work to draw water from the soil.}.

\subsubsection{Resources}
The soil moisture sensor exposes two resources: the \textit{soil\_moisture\_sensor} and the \textit{soil\_moisture\_switch} resources.

The \textbf{soil\_moisture\_sensor} resource is an observable resource that provides to the clients the soil moisture tension acquired by the sensor. The resource not only provides the mere tension to the clients, but it informs if the value is lower or greater than a certain threshold. Hence, the sensor exposes a  \textit{PUT} method, in order to set up the lower or the upper bound for the tension\footnote{For the default range value we used the ones indicated here: https://www.metergroup.com/environment/articles/defining-water-potential/ }.

The change of the bounds is done at step, the user will specify the threshold that he/she wants to change through the CLI: upper or lower. At the server side the request will be processed checking if the value arrived is consistent (e.g., the new value for the lower bound is not greater than the upper bound actual value), after those controls the parameter is updated.

The \textbf{soil\_moisture\_switch} resource is connected to the \textit{isActive} boolean variable, which indicates if the sensor is operating or not. This is done for turning off the temperature sensor when it is raining in order to save energy, since we do not perform any analysis for irrigating when the weather does that for us. For the reason that we want to change the status of the resources based on the rain sensor, it is implemented a \textit{PUT} method for changing the value of the \textit{isActive} variable.

\subsubsection{Data Generation}
Data is generated every \textit{CLOCK\_SECOND} in order to have a rapid simulation. The value for the tension is updated using the following algorithm (the same of to the one used for the temperature):

\begin{lstlisting}
static void soil_moisture_event_handler(void)
{
    if (!isActive) {
        return; // DOES NOTHING SINCE IT IS TURNED OFF
    }
    
    // extimate new tension
    srand(time(NULL));
    int new_soilTension = soilTension;
    int random = rand() % 4; // generate 0, 1, 2, 3
    
    if (random == 0) // 25% of changing the value
        if (random < 2) // decrease
            new_soilTension -= VARIATION;
        else // increase
            new_soilTension += VARIATION;

    // if not equal
    if (new_soilTension != soilTension)
    {
        soilTension = new_soilTension;
        coap_notify_observers(&soil_moisture_sensor);
    }
}
\end{lstlisting}



\subsection{Rain Sensor}
Rain sensor or rain switch is a switching device activated by rainfall. It is used for water conservation since it is connected to the automatic irrigation system, which will cause the system to shut down in the event of rainfall in order to do not waste water and to reduce energy consumption.

\subsubsection{Resource}
The only resource provided by the rain sensor is a value indicating if it is raining or not, named \textbf{isRaining} and stored as a boolean. Since we are only interested when the status of the variable change, we opt to use the observable pattern provided by CoAP in order to minimize the number of interactions with the sensor.

The only possible action is the \textbf{GET} method, which will respond with a text saying \textit{"raining"} or \textit{"not raining"} based on the status of \textit{isRaining}.

\subsubsection{Data Generation}
Data is generated every \textit{CLOCK\_SECOND} in order to have a rapid simulation. The value of \textbf{isRaining} flips (i.e., if it was indicating raining it turns to not raining, and vice versa) with a probability of 10\%. This is done in the \textit{rain\_event\_handler} function in the following way:

\begin{lstlisting}
static void rain_event_handler(void)
{
    // check if raining
    srand(time(NULL));
    int random = rand() % 10; // generate 0, 1, ..., 9
    
    bool new_isRaining = isRaining;
    if (random == 0) // 10% of changing the value
        new_isRaining = !isRaining;

    // if not equal, notify
    if (new_isRaining != isRaining) {
        isRaining = new_isRaining;
        coap_notify_observers(&rain_sensor);
    }
}
\end{lstlisting}

In case the value changes, this is notified to all the subscribers.




\subsection{Tap Actuator}
The tap actuator is the device aim at irrigating the fields.

\subsubsection{Resources}
The tap actuator exposes four resources: the \textit{tap\_intensity}, the \textit{tap\_interval}, \textit{tap\_where\_water} and the \textit{tap\_switch}.

The \textbf{tap\_intensity} resource is related to the \textit{intensity} variable, which indicates the volume of water to provide at each simulation. Since we want to pull water from the aquifer or the reservoir, we made this resource observable in the way of better controlling the \textit{water level sensors} in the simulation. The method implemented for this resource are the \textit{GET} and \textit{PUT} methods. The \textit{GET} method respond to the user with a string containing the value of the intensity and, separated by a space, a character value indicating if the water is taken from the aquifer ('A') or the reservoir('R'). On the other hand, the \textit{PUT} method gives to the user the possibility to set up the intensity value updating the current one.

The \textbf{tap\_interval} resource indicates the period of time (expressed in \textit{CLOCK\_SECOND}) between two activations of the tap. This resource can be retrieved using the \textit{GET} method associated to it, and it can be updated using the \textit{PUT} method. The \textit{PUT} method does not only update the value, but it updates also the intensity based on the following formula.

\begin{equation}\label{eq1}
  \begin{gathered}
    \text{intensity} = \text{intensity} * \frac{\text{new\_interval\_value}}{\text{old\_interval\_value}}
  \end{gathered}
\end{equation}

The \textbf{tap\_where\_water} resource is used by the device in order to decide where to take the water for every irrigation activity. Thus, the resource provide a \textit{PUT} method for changing this value and, for sake of control, we implemented the possibility to retrieve the value through a \textit{GET} method.

The \textbf{tap\_switch} resource works in the same way of the \textbf{temperature\_switch} and \textbf{soil\_moisture\_switch} resources. The meaning of use this type of resource is for not irrigating when raining since it would be a waste.